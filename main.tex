\documentclass{article}

\usepackage{amsmath,amssymb,amsthm}

\usepackage{xcolor}

\usepackage{multirow}
\usepackage{colortbl}

\usepackage{tikz}
\usetikzlibrary{arrows.meta,graphs,graphdrawing,positioning,shapes}
\usegdlibrary{layered,trees}
\tikzset{
	n/.style={rectangle split, rectangle split, rectangle split parts=2,
	draw=black!80!white,fill=black!15,thick,inner sep=3pt,align=center}, % node
	ln/.style={anchor=south}, % label number,
	c/.style={anchor=north,align=center} % comment
}

\usepackage{hyperref}
\hypersetup{colorlinks=true}

\author{Guanyuming He\\ID: 2035573}
\title{INT102 Assignment 2 Submission}
\date{\today}

\begin{document}
	\pagenumbering{gobble}
	\maketitle
	
	\newpage
	\pagenumbering{arabic}
	
\section{Question \arabic{section}}
\subsection{\arabic{subsection})}
\begin{center}
\begin{tabular}{|c|c|c|c|}
\hline
A & G & C & T \\
\hline
1 & 2 & 5 & 3 \\
\hline
\end{tabular}
\end{center}

\subsection{\arabic{subsection})}
The process is shown below:
\begin{center}
\begin{tabular}{|ccccccccc|}
	\hline
	A & G & C & A & A & T & G & A & A \\
	A & T & G & A & A &   &   &   &   \\
	  & A & T & G & A & A &   &   &   \\
	  &   &   &   & A & T & G & A & A \\
	\hline
\end{tabular}
\end{center}
The total number of comparisons is 3 (C $\ne$ G) + 1 (T $\ne$ A) + 5 (match found) = 9.

\section{Question \arabic{section}}
\subsection{\arabic{subsection})}
The process is shown below, assuming that each iteration goes through the edges in the order from $e_1$ to $e_6$.
\begin{center}
\begin{tabular}{|l|c|c|c|c|c|}
\hline
iteration & a & b & c & d & e \\
\hline
before & (a, 0) & (-, $\infty$) & (-, $\infty$) & (-, $\infty$) & (-, $\infty$) \\
\hline
1 finish & (a, 0) & \textcolor{red}{(a, 4)} & \textcolor{red}{(a, 5)} & (-, $\infty$) & (-, $\infty$) \\
\hline
2 until $e_3$ & (a, 0) & (a, 4) & (a, 5) & \textcolor{red}{(b, 14)} & (-, $\infty$) \\
2 until $e_4$ & (a, 0) & (a, 4) & (a, 5) & (b, 14) & \textcolor{red}{(c, -2)} \\
2 finish & (a, 0) & \textcolor{red}{(e, -5)} & (a, 5) & (b, 14) & (c, -2) \\
\hline
3 until $e_3$ & (a, 0) & (e, -5) & (a, 5) & \textcolor{red}{(b, 5)} & (c, -2) \\
3 finish & (a, 0) & (e, -5) & (a, 5) & (b, 5) & (c, -2) \\
\hline
4 finish & (a, 0) & (e, -5) & (a, 5) & (b, 5) & (c, -2) \\
\hline
5 finish & (a, 0) & (e, -5) & (a, 5) & (b, 5) & (c, -2) \\
\hline
\end{tabular}
\end{center}
The shortest path from a to a is \{(a,a)\}. From a to b is \{(a,c), (c,e) (e,b)\}. From a to c is \{(a,c)\}. From a to d is \{(a,c), (c,e), (e,b), (b,d)\}. From a to e is \{(a,c), (c,e)\}.

\section{Question \arabic{section}}
\subsection{\arabic{subsection})}
The table is shown below:
\begin{center}
\begin{tabular}{c|c|c|c|c|c|c|c|}
\multicolumn{1}{c}{ } & \multicolumn{1}{c}{-} & \multicolumn{1}{c}{\textcolor{red}{A}} & \multicolumn{1}{c}{\textcolor{red}{G}} & \multicolumn{1}{c}{C} & \multicolumn{1}{c}{C} & \multicolumn{1}{c}{C} & \multicolumn{1}{c}{\textcolor{red}{T}} \\
\cline{2-8}
- & 0 & 0 & 0 & 0 & 0 & 0 & 0 \\
\cline{2-8}
G & 0 & 0$\uparrow$ & 1$\nwarrow$ & 1$\leftarrow$ & 1$\leftarrow$ & 1$\leftarrow$ & 1$\leftarrow$ \\
\cline{2-2}
\textcolor{red}{A} & 0 & \cellcolor{red!50}1$\nwarrow$ & 1$\uparrow$ & 1$\uparrow$ & 1$\uparrow$ & 1$\uparrow$ & 1$\uparrow$ \\
\cline{2-2}
\textcolor{red}{G} & 0 & 1$\uparrow$ & \cellcolor{red!50}2$\nwarrow$ & \cellcolor{red!50}2$\leftarrow$ & \cellcolor{red!50}2$\leftarrow$ & \cellcolor{red!50}2$\leftarrow$ & \cellcolor{white}2$\leftarrow$ \\
\cline{2-2}
\textcolor{red}{T} & 0 & 1$\uparrow$ & 2$\uparrow$ & 2$\uparrow$ & 2$\uparrow$ & 2$\uparrow$ & \cellcolor{red!50}3$\nwarrow$ \\
\cline{2-8}
\end{tabular}
\end{center}

\subsection{\arabic{subsection})}
According to the table, the longest subsequence is AGT.

\section{Question \arabic{section}}
\subsection{\arabic{subsection})}
\subsubsection{a)}
The table is shown below:
\begin{center}
\begin{tabular}{c|c|c|c|c|c|c|c|}
\multicolumn{1}{c}{ } & \multicolumn{1}{c}{-} & \multicolumn{1}{c}{A} & \multicolumn{1}{c}{G} & \multicolumn{1}{c}{A} & \multicolumn{1}{c}{C} & \multicolumn{1}{c}{C} & \multicolumn{1}{c}{T} \\
\cline{2-8}
- & 0 & \cellcolor{red!50}-1 & -2 & -3 & -4 & -5 & -6 \\
\cline{2-8}
G & \cellcolor{red!50}-1 & -2$\leftarrow\nwarrow$$\uparrow$ & \cellcolor{red!50}0$\nwarrow$ & -1$\leftarrow$ & -2$\leftarrow$ & -3$\leftarrow$ & -4$\leftarrow$ \\
\cline{2-2}
A & -2 & \cellcolor{red!50}0$\nwarrow$ & -1$\leftarrow\uparrow$ & \cellcolor{red!50}1$\nwarrow$ & \cellcolor{red!50}0$\leftarrow$ & \cellcolor{red!50}-1$\leftarrow$ & -2$\leftarrow$ \\
\cline{2-2}
G & -3 & -1$\uparrow$ & \cellcolor{red!50}1$\nwarrow$ & \cellcolor{red!50}0$\leftarrow\uparrow$ & \cellcolor{red!50}-1$\leftarrow\uparrow$ & \cellcolor{red!50}-2$\leftarrow\uparrow$ & -3$\leftarrow\uparrow$ \\
\cline{2-2}
T & -4 & -2$\uparrow$ & 0$\uparrow$ & -1$\leftarrow\uparrow$ & -2$\leftarrow\nwarrow\uparrow$ & -3$\leftarrow\nwarrow\uparrow$ & \cellcolor{red!50}-1$\nwarrow$ \\
\cline{2-8}
\end{tabular}
\end{center}
\subsubsection{b)}
According to the table, an optimal global alignments is:
\begin{center}
\begin{tabular}{ccccccc}
	 A &  G &  A &  C &  C & \_ & T \\
	\_ &  G &  A & \_ & \_ &  G & T
\end{tabular}
\end{center}

\subsection{\arabic{subsection})}
\subsubsection{a)}	
The table is shown below:
\begin{center}
\begin{tabular}{c|c|c|c|c|c|c|c|}
\multicolumn{1}{c}{ } & \multicolumn{1}{c}{-} & \multicolumn{1}{c}{A} & \multicolumn{1}{c}{G} & \multicolumn{1}{c}{A} & \multicolumn{1}{c}{C} & \multicolumn{1}{c}{C} & \multicolumn{1}{c}{T} \\
\cline{2-8}
- & 0 & \cellcolor{red!50}0 & 0 & 0 & 0 & 0 & 0 \\
\cline{2-8}
G & \cellcolor{red!50}0 & 0 & \cellcolor{red!50}1$\nwarrow$ & 0$\leftarrow$ & 0 & 0 & 0 \\
\cline{2-2}
A & 0 & \cellcolor{red!50}1$\nwarrow$ & 0$\leftarrow$ & \cellcolor{red!50}2$\nwarrow$ & 1$\leftarrow$ & 0$\leftarrow$ & 0 \\
\cline{2-2}
G & 0 & 0$\uparrow$ & \cellcolor{red!50}2$\nwarrow$ & 1$\leftarrow$ & 0$\leftarrow\uparrow$ & 0 & 0 \\
\cline{2-2}
T & 0 & 0 & 1$\uparrow$ & 0$\leftarrow\uparrow$ & 0 & 0 & 1$\nwarrow$ \\
\cline{2-8}
\end{tabular}
\end{center}

\subsubsection{b)}
According to the table, an optimal local alignment is:
\begin{tabular}{cc}
G & A \\
G & A
\end{tabular}

\section{Question \arabic{section}}
First fill in the matrix:
\begin{center}
\begin{tabular}{c|ccccc}
  & a & b & c & d & e \\
\hline
a & 0 & 4 & 5 & 2 & 1 \\
b & 4 & 0 & 4 & 3 & 1 \\
c & 5 & 4 & 0 & 1 & 8 \\
d & 2 & 3 & 1 & 0 & 6 \\
e & 1 & 1 & 8 & 6 & 0 \\
\end{tabular}
\end{center}
At the beginning, the lower bound is 
\[
\lceil\frac{1+2}{2} + \frac{1+3}{2} + \frac{1+4}{2} + \frac{1+2}{2} + \frac{1+1}{2}\rceil = 9
\]

It is sufficient to only consider tours starting from vertex a. In addition, since it is an undirected graph, a requirement is made that b must be travelled before c, under which it will still be enough to get the correct result. The calculation process is shown below:
\begin{center}
\begin{tikzpicture}[tree layout,font=\tiny,level distance=1.4cm, sibling sep=.6em, sibling distance=1.1cm]
{
\node (a) [n] {0: a\nodepart{two}$lb = 9$};

\node (ab) [n] {1: a,b\nodepart{two}$lb = 10$};
\node (ac) [n] {2: a,c\nodepart{two}$\times$\\b after c};
\node (ad) [n] {3: a,d\nodepart{two}$lb = 9$}; 
\node (ae) [n] {4: a,e\nodepart{two}$lb = 9$}; 

\node (adb) [n] {5: a,d,b\nodepart{two}$lb = 10$}; 
\node (adc) [n] {6: a,d,c\nodepart{two}$\times$\\b after c}; 
\node (ade) [n] {7: a,d,e\nodepart{two}$lb = 14$}; 

\node (adbce) [n] {8: a,d,b,c,\\(e,a)\nodepart{two}$I = 18$};
\node (adbec) [n] {9: a,d,b,e,\\(c,a)\nodepart{two}$I = 19$};

\node (adebc) [n] {10: a,d,e,b,\\(c,a)\nodepart{two}$I = 18$};
\node (adecb) [n] {11: a,d,e,c,\\(b,a)\nodepart{two}$\times$\\b after c};

\node (aeb) [n] {12: a,e,b\nodepart{two}$lb = 9$}; 
\node (aec) [n] {13: a,e,c\nodepart{two}$\times$\\b after c}; 
\node (aed) [n] {14: a,e,d\nodepart{two}$lb = 14$\\$\times$\\$> I_{15} = 9$};

\node (aebcd) [n] {15: a,e,b,c,\\(d,a)\nodepart{two}$I = 9$};
\node (aebdc) [n] {16: a,e,b,d,\\(c,a)\nodepart{two}$I = 11$};
	
\draw 	(a) edge (ab)
			edge (ac)
			edge (ad)
			edge (ae)
		(ad) edge (adb)
			 edge (adc)
			 edge (ade)
		(adb) edge (adbce)
			  edge (adbec)
		(ade) edge (adebc)
			  edge (adecb)
		(ae) edge (aeb)
			 edge (aec)
			 edge (aed)
		(aeb) edge (aebcd)
			  edge (aebdc);
};
\end{tikzpicture}
\end{center}

According to the process, the optimal tour is: $a \to e \to b \to c \to d \to a$.

\section{Question \arabic{section}}
\begin{enumerate}
	\item Contradicts to the knowledge as $\operatorname{NP} \ne \operatorname{NPC}$.
	\item Although currently most computer scientists believe that $\operatorname{P} \ne \operatorname{NP}$, it is still possible.
	\item It is not clear whether $\operatorname{NP} = \operatorname{P} \cup \operatorname{NPC}$ or not.
	\item Contradicts to the knowledge. If $\operatorname{P} \cap \operatorname{NPC} \ne \{\}$, then for some problem $p \in \operatorname{NPC}$, $p$ is solvable in polynomial time. However, as every problem in $\operatorname{NP}$ can be polynomially reduced to some problem in $\operatorname{NPC}$, then every problem in it has to be polynomially solvable, which means $\operatorname{P} = \operatorname{NP}$, a contradiction.
	\item It is possible.
\end{enumerate}

\section{Question \arabic{section}}
Yes, I completed the coursework individually.

\end{document}


