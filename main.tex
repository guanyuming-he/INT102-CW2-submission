\documentclass{article}

\usepackage{amsmath}
\usepackage{amssymb}
\usepackage{amsthm}

\usepackage{xcolor}

\usepackage{multirow}
\usepackage{colortbl,hhline}
\usepackage{tikz}

\usepackage{hyperref}
\hypersetup{colorlinks=true}

\author{Guanyuming He\\ID: 2035573}
\title{INT102 Assignment 2 Submission}
\date{\today}

\begin{document}
	\pagenumbering{gobble}
	\maketitle
	
	\newpage
	\pagenumbering{arabic}
	
\section{Question \arabic{section}}
\subsection{\arabic{subsection})}
\begin{center}
\begin{tabular}{|c|c|c|c|}
\hline
A & G & C & T \\
\hline
1 & 2 & 5 & 3 \\
\hline
\end{tabular}
\end{center}

\subsection{\arabic{subsection})}
The process is shown below:
\begin{center}
\begin{tabular}{|ccccccccc|}
	\hline
	A & G & C & A & A & T & G & A & A \\
	A & T & G & A & A &   &   &   &   \\
	  & A & T & G & A & A &   &   &   \\
	  &   &   &   & A & T & G & A & A \\
	\hline
\end{tabular}
\end{center}
The total number of comparisons is 3 (C $\ne$ G) + 1 (T $\ne$ A) + 5 (match found) = 9.

\section{Question \arabic{section}}
\subsection{\arabic{subsection})}
The process is shown below, assuming that each iteration goes through the edges in the order from $e_1$ to $e_6$.
\begin{center}
\begin{tabular}{|l|c|c|c|c|c|}
\hline
iteration & a & b & c & d & e \\
\hline
before & (a, 0) & (-, $\infty$) & (-, $\infty$) & (-, $\infty$) & (-, $\infty$) \\
\hline
1 finish & (a, 0) & \textcolor{red}{(a, 4)} & \textcolor{red}{(a, 5)} & (-, $\infty$) & (-, $\infty$) \\
\hline
2 until $e_3$ & (a, 0) & (a, 4) & (a, 5) & \textcolor{red}{(b, 14)} & (-, $\infty$) \\
2 until $e_4$ & (a, 0) & (a, 4) & (a, 5) & (b, 14) & \textcolor{red}{(c, -2)} \\
2 finish & (a, 0) & \textcolor{red}{(e, -5)} & (a, 5) & (b, 14) & (c, -2) \\
\hline
3 until $e_3$ & (a, 0) & (e, -5) & (a, 5) & \textcolor{red}{(b, 5)} & (c, -2) \\
3 finish & (a, 0) & (e, -5) & (a, 5) & (b, 5) & (c, -2) \\
\hline
4 finish & (a, 0) & (e, -5) & (a, 5) & (b, 5) & (c, -2) \\
\hline
5 finish & (a, 0) & (e, -5) & (a, 5) & (b, 5) & (c, -2) \\
\hline
\end{tabular}
\end{center}
The shortest path from a to a is \{(a,a)\}. From a to b is \{(a,c), (c,e) (e,b)\}. From a to c is \{(a,c)\}. From a to d is \{(a,c), (c,e), (e,b), (b,d)\}. From a to e is \{(a,c), (c,e)\}.

\section{Question \arabic{section}}
\subsection{\arabic{subsection})}
The table is shown below:\\
\begin{center}
\begin{tabular}{c|c|c|c|c|c|c|c|}
\multicolumn{1}{c}{ } & \multicolumn{1}{c}{-} & \multicolumn{1}{c}{\textcolor{red}{A}} & \multicolumn{1}{c}{\textcolor{red}{G}} & \multicolumn{1}{c}{C} & \multicolumn{1}{c}{C} & \multicolumn{1}{c}{C} & \multicolumn{1}{c}{\textcolor{red}{T}} \\
\cline{2-8}
- & 0 & 0 & 0 & 0 & 0 & 0 & 0 \\
\cline{2-8}
G & 0 & 0$\uparrow$ & 1$\nwarrow$ & 1$\leftarrow$ & 1$\leftarrow$ & 1$\leftarrow$ & 1$\leftarrow$ \\
\cline{2-2}
\textcolor{red}{A} & 0 & \cellcolor{red!50}1$\nwarrow$ & 1$\uparrow$ & 1$\uparrow$ & 1$\uparrow$ & 1$\uparrow$ & 1$\uparrow$ \\
\cline{2-2}
\textcolor{red}{G} & 0 & 1$\uparrow$ & \cellcolor{red!50}2$\nwarrow$ & \cellcolor{red!50}2$\leftarrow$ & \cellcolor{red!50}2$\leftarrow$ & \cellcolor{red!50}2$\leftarrow$ & \cellcolor{white}2$\leftarrow$ \\
\cline{2-2}
\textcolor{red}{T} & 0 & 1$\uparrow$ & 2$\uparrow$ & 2$\uparrow$ & 2$\uparrow$ & 2$\uparrow$ & \cellcolor{red!50}3$\nwarrow$ \\
\cline{2-8}
\end{tabular}
\end{center}

\subsection{\arabic{subsection})}
According to the table, the longest subsequence is AGT.

\section{Question \arabic{section}}
\subsection{\arabic{subsection})}
\subsubsection{a)}
The table is shown below:\\
\begin{center}
\begin{tabular}{c|c|c|c|c|c|c|c|}
\multicolumn{1}{c}{ } & \multicolumn{1}{c}{-} & \multicolumn{1}{c}{A} & \multicolumn{1}{c}{G} & \multicolumn{1}{c}{A} & \multicolumn{1}{c}{C} & \multicolumn{1}{c}{C} & \multicolumn{1}{c}{T} \\
\cline{2-8}
- & 0 & \cellcolor{red!50}-1 & -2 & -3 & -4 & -5 & -6 \\
\cline{2-8}
G & \cellcolor{red!50}-1 & -2$\leftarrow\nwarrow$$\uparrow$ & \cellcolor{red!50}0$\nwarrow$ & -1$\leftarrow$ & -2$\leftarrow$ & -3$\leftarrow$ & -4$\leftarrow$ \\
\cline{2-2}
A & -2 & \cellcolor{red!50}0$\nwarrow$ & -1$\leftarrow\uparrow$ & \cellcolor{red!50}1$\nwarrow$ & \cellcolor{red!50}0$\leftarrow$ & \cellcolor{red!50}-1$\leftarrow$ & -2$\leftarrow$ \\
\cline{2-2}
G & -3 & -1$\uparrow$ & \cellcolor{red!50}1$\nwarrow$ & \cellcolor{red!50}0$\leftarrow\uparrow$ & \cellcolor{red!50}-1$\leftarrow\uparrow$ & \cellcolor{red!50}-2$\leftarrow\uparrow$ & -3$\leftarrow\uparrow$ \\
\cline{2-2}
T & -4 & -2$\uparrow$ & 0$\uparrow$ & -1$\leftarrow\uparrow$ & -2$\leftarrow\nwarrow\uparrow$ & -3$\leftarrow\nwarrow\uparrow$ & \cellcolor{red!50}-1$\nwarrow$ \\
\cline{2-8}
\end{tabular}
\end{center}
\subsubsection{b)}
According to the table, an optimal global alignments is:\\
\begin{center}
\begin{tabular}{ccccccc}
	 A &  G &  A &  C &  C & \_ & T \\
	\_ &  G &  A & \_ & \_ &  G & T
\end{tabular}
\end{center}

\subsection{\arabic{subsection})}
\subsubsection{a)}	
The table is shown below:\\
\begin{center}
\begin{tabular}{c|c|c|c|c|c|c|c|}
\multicolumn{1}{c}{ } & \multicolumn{1}{c}{-} & \multicolumn{1}{c}{A} & \multicolumn{1}{c}{G} & \multicolumn{1}{c}{A} & \multicolumn{1}{c}{C} & \multicolumn{1}{c}{C} & \multicolumn{1}{c}{T} \\
\cline{2-8}
- & 0 & \cellcolor{red!50}0 & 0 & 0 & 0 & 0 & 0 \\
\cline{2-8}
G & \cellcolor{red!50}0 & 0 & \cellcolor{red!50}1$\nwarrow$ & 0$\leftarrow$ & 0 & 0 & 0 \\
\cline{2-2}
A & 0 & \cellcolor{red!50}1$\nwarrow$ & 0$\leftarrow$ & \cellcolor{red!50}2$\nwarrow$ & 1$\leftarrow$ & 0$\leftarrow$ & 0 \\
\cline{2-2}
G & 0 & 0$\uparrow$ & \cellcolor{red!50}2$\nwarrow$ & 1$\leftarrow$ & 0$\leftarrow\uparrow$ & 0 & 0 \\
\cline{2-2}
T & 0 & 0 & 1$\uparrow$ & 0$\leftarrow\uparrow$ & 0 & 0 & 1$\nwarrow$ \\
\cline{2-8}
\end{tabular}
\end{center}

\subsubsection{b)}
According to the table, an optimal local alignment is:\\
\begin{center}
\begin{tabular}{cc}
G & A \\
G & A
\end{tabular}
\end{center}

\end{document}


